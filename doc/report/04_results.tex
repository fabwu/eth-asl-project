\section{Results}\label{sec:exp}

In the first part of this section we describe the benchmark infrastructure and the
images we used for the measurements. Then we analyze how different compilers and flags  
affect the performance of our code. At the end we discuss each optimization and compare
them in different plots.

\mypar{Experimental setup} All benchmarks and tests were conducted on an Intel
i7-8650U processor with \textit{Intel Turbo Boost} disabled, running at 1.9
GHz. The CPU has a 4$\times$32 KB 8-way associative L1 cache, a 4$\times$256 KB
4-way associative L2 cache and 4$\times$2 MB 16-way associative L3 cache
\cite{intel-opt-manual}. 

We used the peak signal-to-noise ratio (PSNR) to ensure that our baseline implementation
compresses the image correctly. This metric is widely used to compare an image with its
compressed version. Typical values for the PSNR range from 25 to 
50 dB (higher is better). The same metric was used to verify that the optimizations 
produced the same result as the baseline.

The PSNR of the compressed image depends on the maximum depth of the quadtree and the error
threshold. We picked a maximum depth of $m=7$ and an error threshold of $\epsilon=300$ for
all our experiments. These parameters produced good quality in a reasonable amount of time.

We chose to benchmark our algorithm with a challenging image depicting a lioness
with its cub \cite{lions}.
While some parts of the image, for example the
background, are easy to compress, other parts such as the fur contains lots of details,
which are harder to compress. Figure \ref{fig:lions} shows the output of the
vectorized code with $\epsilon=100$ and 3 decompression iterations.

\begin{figure}
  \centering
  \includegraphics[page=1, width=.45\textwidth]{lion_512_51_e100}
  \caption{Decompressed Image}
  \label{fig:lions}
\end{figure}

\mypar{Compiler Flags} Several benchmarks were conducted comparing the achieved
performance using different compiler flags for the \textit{GNU Compiler
  Collection (gcc)} version 9.3.0 and the \textit{Intel C++ Compiler (icc)}
version 19.1.1.217. For all tests \texttt{-march=native} was set.

For both the scalar and the vectorized versions the flag \texttt{-O1} increased
performance significantly. While \texttt{-O2} did improve the scalar
implementation (figure \ref{fig:perf_scal}), it did not make a difference for
the vectorized version (figure \ref{fig:perf_vec}). Neither \textsc{-O3} nor
\texttt{-Ofast} were able increase performance for both code versions.

\begin{figure}
  \begin{subfigure}[t]{\linewidth}
    \centering
    \includegraphics[page=1, width=\linewidth]{performance_scalar_opts}
    \caption{Scalar Version}
    \label{fig:perf_scal}
  \end{subfigure}
  \begin{subfigure}[b]{\linewidth}
    \centering
    \includegraphics[page=1, width=\linewidth]{performance_vectorized_opts}
    \caption{Vectorized Version}
    \label{fig:perf_vec}
  \end{subfigure}
  \caption{Optimization Flags}
\end{figure}

%\notepascal{-O2 is hidden behind -O3 in (a) :/}

Intel's compiler was not able to outperform gcc but at least for the vectorized
version it managed to keep up for the larger images as shown in figure
\ref{fig:perf_gcc_vs_icc}.

\begin{figure}
  \centering
  \includegraphics[page=1, width=\linewidth]{performance_gcc_vs_icc}
  \caption{GCC vs. ICC}
  \label{fig:perf_gcc_vs_icc}
\end{figure}

We used gcc with \verb|-O3 -NDEBUG -march=native| for all benchmarks because the
Intel compiler or different flags didn't improve performance significantly.
\notejonas{Was heisst das? Einfach nur für die Performance Benchmarks?}

\mypar{Performance} The plot in figure \ref{fig:perf} shows the performance of all
experiments.

In our first experiment we compared the baseline with a popular C\texttt{++} 
implementation from GitHub \cite{github-cpp}. This comparison ensures that our baseline
achieves a reasonable performance. The two lines at the bottom of the plot
show that our baseline performed slightly better than the C\texttt{++} code. The values for large
images are not shown in the plot because the measurements took too much time.
\notejonas{Plot fehlt noch?}

\begin{figure}
  \begin{subfigure}[t]{\linewidth}
    \centering
    \includegraphics[page=1, width=\linewidth]{performance}
    \caption{Performance}
    \label{fig:perf}
  \end{subfigure}
  \begin{subfigure}[t]{\linewidth}
    \centering
    \includegraphics[page=1, width=\linewidth]{runtime}
    \caption{Runtime}
    \label{fig:runtime}
  \end{subfigure}
  \caption{Comparison of Major Implementations}
\end{figure}

The next line shows the performance of the scalar optimized code. Several precomputations, a better
memory layout and ILP made this version 3x faster than the baseline. The optimizations become even 
more apparent when we compare the runtime as in figure \ref{fig:runtime}. The plot shows that the runtime
decreased by several orders of magnitude and we were finally able to compress larger images in a reasonable 
amount of time.

%\notefabian{Warum haben wir nur 50\% scalar peak performance?}

As described in section \ref{sec:yourmethod} the initial attempt in vectorizing the code did not lead
to expected performance improvements. In fact, it performed just slightly better than the scalar optimized code.
We suspected that the gathering instructions, which are necessary for the rotations, are responsible for poor performance. 
To verify this assumption, we removed the column wise access to the image i.e 90/270 degree rotations from the scalar optimized and the vectorized version.
The result of this experiment is shown in figure \ref{fig:perf_40_41} and one can clearly see how the vectorized
code outperforms the scalar optimized version.

\begin{figure}
  \centering
  \includegraphics[page=1, width=.45\textwidth]{performance_40_41}
  \caption{Performance Plot without 90/270 degree rotations}
  \label{fig:perf_40_41}
\end{figure}

After this experiment we improved the column wise access by rotating the range 
block instead of the domain block for the 90/270 degree rotations. The improved 
vectorized implementation has a performance roughly eight times as high as the 
baseline and about twice the performance of the scalar optimized version. The 
runtime was also reduced especially for large images.

