% IEEE standard conference template; to be used with:
%   spconf.sty  - LaTeX style file, and
%   IEEEbib.bst - IEEE bibliography style file.
% --------------------------------------------------------------------------

\documentclass[letterpaper]{article}

% custom setup
% Package includes
\usepackage{amsthm}
\usepackage{amsmath}
\usepackage{spconf,amssymb,graphicx}

% Definitions
\newcommand{\R}[0]{\mathbb{R}}
\newcommand{\C}[0]{\mathbb{C}}
\newcommand{\mypar}[1]{{\bf #1.}}
\DeclareMathOperator*{\argmin}{arg\,min}

% Folder for images
\graphicspath{{images/}}

% set pdf attributes
\usepackage[pdftex,
pdftitle={Fractal Image Compression},
pdfauthor={Jonas Hansen, Pascal Huber, Fabian Wuethrich},
pdfkeywords={ETH, Fastcode, ASL, Fractal Image Compression}
]{hyperref}

\newtheorem{definition}{Definition}
\newtheorem{theorem}{Theorem}

% Inline notes with one color per person
\usepackage{todonotes}
\newcommand{\notefabian}[1]{\todo[inline,color=green!20]{\textbf{F:} #1}}
\newcommand{\notejonas}[1]{\todo[inline,color=blue!20]{\textbf{J:} #1}}
\newcommand{\notepascal}[1]{\todo[inline,color=red!20]{\textbf{P:} #1}}

% Algorithm stuff
\usepackage{algorithm}
\usepackage[noend]{algpseudocode}
\usepackage{algorithmicx}
\algnewcommand{\NULL}{\textsc{null}}
\algrenewcommand\textproc{\texttt}

% less margins around captions
% TODO: fix spacing after/before images/captions
% \setlength{\belowcaptionskip}{20pt}
% \addtolength{\textfloatsep}{-2.2in}
% \setlength{\intextsep}{10pt plus 2pt minus 2pt}

% don't highlight links
\usepackage{xcolor}
\hypersetup{
    colorlinks,
    linkcolor={red!50!black},
    citecolor={blue!50!black},
    urlcolor={blue!50!black}
}

% When referencing an image and clicking on the link, jump to the top of the
% image and not to the caption below the image.
\usepackage{hyperref}
\usepackage[all]{hypcap}

% for subfigures
\usepackage{caption}
\usepackage{subcaption}

% space efficient itemize
\usepackage{enumitem}


% set to true or false to enable or disable notes
\newboolean{shownotes}
\setboolean{shownotes}{false}

% Title.
% ------
\title{Performance Optimized Fractal Image Compression\\with Quadtree Partitioning}
%
% Single address.
% ---------------
\name{Jonas Hansen, Pascal Huber, Fabian Wüthrich}
\address{Department of Computer Science\\ ETH Zurich, Switzerland}

% For example:
% ------------
%\address{School\\
%		 Department\\
%		 Address}
%
% Two addresses (uncomment and modify for two-address case).
% ----------------------------------------------------------
%\twoauthors
%  {A. Author-one, B. Author-two\sthanks{Thanks to XYZ agency for funding.}}
%		 {School A-B\\
%		 Department A-B\\
%		 Address A-B}
%  {C. Author-three, D. Author-four\sthanks{The fourth author performed the work
%		 while at ...}}
%		 {School C-D\\
%		 Department C-D\\
%		 Address C-D}
%

\begin{document}
%\ninept
%
\maketitle
%
%

% space efficient align
\setlength{\abovedisplayskip}{1.5ex}
\setlength{\belowdisplayskip}{1.5ex}
% \setlength{\abovedisplayshortskip}{0pt}
% \setlength{\belowdisplayshortskip}{0pt}

\begin{abstract}
  Fractal image compression is a lossy image compression method, which yields
  good compression ratios and quality at the cost of high encoding times. We
  implemented the algorithm from scratch using quadtree partitioning and
  exhaustive search for self-similarity in the image.

  This baseline implementation was analyzed using the roofline model and
  profiling to identify bottlenecks. With these insights we implemented
  precomputations and removed complex data structures to improve performance. In
  addition, we optimized the code for instruction level parallelism
  for a better pipeline utilization. Then we vectorized the code
  using AVX2 intrinsics and changed the memory layout to avoid expensive
  gathering instructions. At last, different compiler flags for GCC and ICC were
  tested.

  The optimized version without vectorization led to a 4x increase in
  performance whereas a speedup of up to 8x was observed with SIMD.
  Additionally, the runtime of the compression was reduced significantly.

  \notejonas{Der letzte Abschnitt ist jetzt fast genau der gleiche wie in der
    Conclusion. Eventuell hier entfernen oder nur perf. speedup erwähnen?}
\end{abstract}

\section{Introduction}\label{sec:intro}

Human perception relies heavily upon visual information, particularly images.
This medium is very powerful in gaining and conveying \textit{ideas}, memories,
and emotions. No one would doubt the saying ``A picture is worth a thousand
words''. Especially with the widespread use of the internet and the advent of
social media the ability to share and store images efficiently is crucial.

A practical way to store images without occupying too much space is to use lossy
image compression which decreases the image quality in places where a
high-quality image is not perceivable or where an image of lower quality is not
disruptive. JPEG, whose basic building block is the discrete cosine transform,
is undoubtedly the most widely used lossy compression scheme. With a good
compression quality and fast compression times, JPEG hits a sweet spot for many
practical applications.

Fractal image compression is another method for lossy image compression. The
main idea is to compress an image by exposing self-similarity which can be
observed in many places in nature (e.g. fir cones or romanesco broccoli). 
Contrary to JPEG, the underlying theoretical construct of a fractal compression
scheme is that of an iterated function system (IFS). 

This approach yields great
compression results in terms of compression size and quality \cite{fisher2012}.
However, it is computationally expensive to encode images because the algorithm
involves an exhaustive search over different regions of the image with many
numerical computations to find self-similarity. Optimizing these computations is
therefore crucial for an efficient implementation of the algorithm.

\mypar{Related work} The most widely known practical fractal compression scheme
was developed and patented by Michael Barnsley and Alan Sloan in 1987. They
published a paper about their work in 1989 \cite{barnsley1989fractal}. Barnsleys
graduate student Arnaud Jacquin was the first who implemented a practical
version of it in 1992 in his PhD thesis \cite{jacquin1990fractal}. Numerous
improvements and variations have then been developed to this original approach,
e.g. archetype classification (\cite{jacobs1992image}, \cite{boss1991studies})
which decreases the exhaustive search space for self-similarity. Yuval Fisher
published a book in 1995 with a detailed description of various fractal schemes
and an elaborate list of optimizations \cite{fisher2012}.

\mypar{Contribution} Based on the explanations of Fisher in \cite{fisher2012}
and an open-source implementation written in Python \cite{github-python} and C++
\cite{github-cpp}, we implemented our own fractal image compression scheme with
quadtree partitioning and exhaustive self-similarity search.

In our work, we did not focus on algorithmic optimizations, but optimized the
computations of the given algorithm.

\notepascal{sit der letzte Satz notwendig?}

\section{Background on the Algorithm/Application}\label{sec:background}

Give a short, self-contained summary of necessary
background information on the algorithm or application that you then later optimize including a cost analysis.

For example, assume you present an
implementation of FFT algorithms. You could organize into DFT
definition, FFTs considered, and cost analysis. The goal of the
background section is to make the paper self-contained for an audience
as large as possible. As in every section
you start with a very brief overview of the section. Here it could be as follows: In this section
we formally define the discrete Fourier transform, introduce the algorithms we use
and perform a cost analysis.

\mypar{Discrete Fourier Transform}
Precisely define the transform so I understand it even if I have never
seen it before.

\mypar{Fast Fourier Transforms}
Explain the algorithm you use.

\mypar{Cost Analysis}
First define you cost measure (what you count) and then compute or determine on other ways the
cost as explained in class. In the end you will likely consolidate it into one number (e.g., adds/mults/comparisons) but be aware of major imbalances as they affect the peak performance..

Also state what is known about the complexity (asymptotic usually)
about your problem (including citations).
\section{Methodology}\label{sec:yourmethod}

In this section we define the scope of our implementation, describe the baseline
implementation and its bottlenecks and outline the steps which were followed to
increase the performance. At the end the remaining performance blockers.

\mypar{Basic implementation}
The baseline is written from scratch in C using an iterative quadtree approach.

Algorithm \ref{alg:baseline} illustrates the compression of the algorithm, whose inputs are the image (row-wise array of doubles) of size $S \times S$,
the max quadtree depth $m$ and the error threshold $\epsilon$.\\
The function \textsc{partition($image$,$s$)} partitions the image into contiguous non-overlapping blocks of size $s \times s$.
The function \textsc{quad($R_i$, $s$)} takes a range block of size $s \times s$ and partitions it into 4 smaller range blocks.
The function \textsc{compute($image$,$R_i$, $D_i$)} computes a transformation with and the resulting RMS according to section \ref{sec:background}. 
Note that the function implicitly rotates the domain block, i.e. the function looks at the four possible locations and returns the transformation with the smallest error.
\begin{algorithm}
\caption{Compression using iterative quadtree}\label{alg:baseline}
\hspace*{\algorithmicindent} \textbf{Input:} $img$ (image of size $S \times S$), $\epsilon$ (RMS threshold) \\
\hspace*{\algorithmicindent} \textbf{Output:} $\boldsymbol{T}$ (set of computed transformations)
\begin{algorithmic}[1]
  \State $\boldsymbol{T} \gets \{\}$ \Comment{Learned transformations} 
  \State $\boldsymbol{R} \gets \Call{partition}{img, S/2}$ \Comment{Initial range blocks} 
    \For{$c=1..m$} \Comment{$c$ is the current quadtree depth}
        \State $\boldsymbol{D} \gets \Call{partition}{img, S/2^{c-1}}$
        \For{$R_i \in \boldsymbol{R}$}
            \State $err_i \gets \infty, T_i \gets \NULL$
            \For{$D_i \in \boldsymbol{D}$}
              \State $T_x, err_x \gets $ \Call{compute}{$img$, $D_i$, $R_i$}
              \If{$err_x < err_i$}
                \State $T_i \gets T_x$
                \State $err_i \gets err_x$
              \EndIf
            \EndFor
        \EndFor
        \State $\boldsymbol{R} \gets \boldsymbol{R} \backslash  R_i$ \Comment{Remove $R_i$ from $\boldsymbol{R}$}
        \If{$err_i > \epsilon$}
          \State $\boldsymbol{R} \gets \boldsymbol{R} \cup  \Call{quad}{R_i, S/2^c}$
        \Else
          \State $\boldsymbol{T} \gets \boldsymbol{T} \cup \{T_i\}$
        \EndIf
    \EndFor
\end{algorithmic}
\end{algorithm}

In order to see if our baseline achieves a reasonable performance, we compared it with
a popular C\texttt{++} implementation from GitHub \cite{github-cpp}. We used the same infrastructure to benchmark
the C\texttt{++} code and our baseline performed slightly better (see !!!ADD LINE TO PERFORMANCEPLOT!!!).

\mypar{Scope} For our implementation only square grayscale images with a power
of two width and height were considered. The reason for the restriction on the
size is that it simplifies \textit{Quadtree Partitioning}. If the width of a
range block is a power of two and no suitable transformation is found for it,
then the four new range blocks created by dividing the original range block into
4 equal parts will also have a power of two width. Colored images can be
compressed and decompressed in the same manner as grayscale images.

\notepascal{are we sure about the color stuff?}

\mypar{Roofline} We used the roofline model \cite{applying-roofline} to see whether the algorithm is memory or
compute bound. The peak performance $\pi$ was determined by counting the dispatch
ports specified in the manual \cite{intel-opt-manual}. We distinguish between scalar and vectorized peak
performance. The bandwidth $\beta$ was measured with the STREAM benchmark \cite{stream}. We verified both the peak
performance and the bandwidth with the Empirical Roofline Tool \cite{ert}, which confirmed the values.

For the program model, we need to measure the work $W$, the runtime $T$ and the data move movement $Q$ of
the algorithm. As the decisions made by the quadtree algorithm differ from image to image, the program
model depends not only on the size but also on the content of an image. Therefore, we define the three 
quantities as $W=W(S, img)$, $T=T(S, img)$ and $Q=Q(S, img)$, where $S$ and $img$ are define as in 
\ref{alg:baseline}. We measured $W$ by instrumenting the code according to the cost metric \ref{eq:cost}.
For measuring $T$, we used the \texttt{RDTSC} instruction available on all x86 architecture and disabled
Turbo Boost to guarantee an integer runtime. The measurement of $Q$ is the most challenging and requires
performance counters which are not supported by our hardware. Thus, we assumed that the image has to be
loaded at least once so we derived a very optimistic lower bound of $Q \geq 8 \cdot S^2$.

To place the baseline in the roofline plot, we calculated the \textit{operational intensity}
$$
I=\frac{W}{Q}
$$
and the \textit{performance}
$$P=\frac{W}{T}$$
of our implementation. Figure \ref{fig:roofline} shows the 
roofline plot. One can see that the baseline is inherently compute bound and we have good potential for
performance improvement as the baseline runs only at 12.5\% of scalar peak performance. Next, we use 
profiling to find the performance bottlenecks of the code.

\begin{figure}
  \includegraphics[page=1, width=.45\textwidth]{roofline}
  \caption{Roofline plot for the baseline and various performance optimizations}
  \label{fig:roofline}
\end{figure}


\mypar{Profiling} With Valgrind's callgrind and cachegrind

\mypar{Bandwidth and Data Transfer}
\mypar{Scalar Optimizations}

\mypar{Vector Optimizations}

\mypar{Block Rotations}

\mypar{Strided Access}

\section{Results}\label{sec:exp}

In the first part of this section, we describe the benchmark infrastructure and the
images we used for the measurements. Then we analyze how different compilers and flags
affect the performance of our code. In the end, we discuss each optimization and compare
them in different plots.

\mypar{Experimental setup} All benchmarks and tests were conducted on an Intel
Core i7-8650U processor with \textit{Intel Turbo Boost} disabled, running at 1.9
GHz. The CPU has a 4$\times$32 KB 8-way associative L1 cache, a 4$\times$256 KB
4-way associative L2 cache and 4$\times$2 MB 16-way associative L3 cache
\cite{intel-opt-manual}.

We used the peak signal-to-noise ratio (PSNR) to ensure that our baseline implementation
compresses the image correctly. This metric is widely applied to compare an image with its
compressed version. Typical values for the PSNR range from 25 to
50 dB (higher is better). The same metric was used to verify that the optimizations
produced the same result as the baseline.

The PSNR of the compressed image depends on the maximum depth $m$ of the quadtree and the error
threshold $\epsilon$. We set $m=7$ and $\epsilon=300$ for all our experiments. These parameters
produced an image of good quality in a reasonable amount of time.

We chose to benchmark our algorithm with a challenging image depicting a lioness
with its cub \cite{lions}.
While some parts of the image, for example the
background, are easy to compress, other parts such as the fur contain lots of details,
which are harder to compress. Figure \ref{fig:lions} shows the output of the
vectorized code with $\epsilon=100$ and 3 decompression iterations.

\begin{figure}[H]
  \centering
  \includegraphics[page=1, width=.45\textwidth]{lion_512_51_e100}
  \caption{Decompressed Image}\label{fig:lions}
\end{figure}

\mypar{Compiler Flags} Several benchmarks were conducted comparing the achieved
performance using different compiler flags for the \textit{GNU Compiler
  Collection (gcc)} version 9.3.0 and the \textit{Intel C++ Compiler (icc)}
version 19.1.1.217. For all tests \texttt{-march=native} was set.

For both the scalar and the vectorized versions the flag \texttt{-O1} increased
performance significantly. While \texttt{-O2} did improve the scalar
implementation (figure \ref{fig:perf_scal}), it did not make a difference for
the vectorized version (figure \ref{fig:perf_vec}). Neither \texttt{-O3} nor
\texttt{-Ofast} were able increase performance for both code versions.

\begin{figure}[H]
  \centering
  \includegraphics[page=1, width=\linewidth]{performance_scalar_opts}
  \caption{Performance of the four major GCC optimization flags for the scalar
    implementation}\label{fig:perf_scal}
\end{figure}

\begin{figure}[H]
  \centering
  \includegraphics[page=1, width=\linewidth]{performance_vectorized_opts}
  \caption{Performance of the four major GCC optimization flags for the vectorized implementation}\label{fig:perf_vec}
\end{figure}

Intel's compiler was not able to outperform gcc but at least for the vectorized
version it managed to keep up for the larger images as shown in figure
\ref{fig:perf_gcc_vs_icc}.

\begin{figure}[H]
  \centering
  \includegraphics[page=1, width=\linewidth]{performance_gcc_vs_icc}
  \caption{Performance comparison between GCC and ICC with the best scalar and
    SIMD implementation}\label{fig:perf_gcc_vs_icc}
\end{figure}

Because the Intel compiler with various compiler flags did not lead to
improvements, we used gcc with the flags \texttt{-march=native -O3} which
performed best for all the upcoming benchmarks.

\mypar{Performance} The plot in figure~\ref{fig:perf} shows the performance of
our major implementations. As a first result we observe that our baseline
implementation achieves an equal or slightly better performance than the
open-source reference C\texttt{++} implementation~\cite{github-cpp}. Values for
large images are not shown in the plot because the measurements took too much
time.

The optimized scalar implementation which includes precomputations, a better
memory layout and ILP is four times faster than the baseline. The optimizations
become even more apparent when we compare the runtime as in figure
\ref{fig:runtime}. The plot shows that the runtime decreased by several orders
of magnitude and we were finally able to compress larger images in a reasonable
amount of time.

\begin{figure}[H]
  \centering
  \includegraphics[page=1, width=\linewidth]{performance_major_versions}
  \caption{Performance of our three major implementations and a reference
    implementation from GitHub written in C\texttt{++}}\label{fig:perf}
\end{figure}


\begin{figure}[H]
  \centering
  \includegraphics[page=1, width=\linewidth]{runtime_major_versions}
  \caption{Runtime of our three major implementations and a reference
    implementation from GitHub written in C\texttt{++}}\label{fig:runtime}
\end{figure}

As described in section \ref{sec:yourmethod} the initial attempt in vectorizing
the code did not lead to the expected performance improvements. In fact, it
performed just slightly better than the scalar optimized code. We suspected that
the gathering instructions, which are necessary for the rotations, are
responsible for poor performance. To verify this assumption, we removed the
column-wise access to the image i.e. 90/270 degree rotations from the scalar
optimized and the vectorized version. The result of this experiment is shown in
figure \ref{fig:perf_40_41} and one can clearly see how the vectorized code
outperforms the scalar optimized version.

\begin{figure}[H]
  \centering
  \includegraphics[page=1, width=\linewidth]{performance_norot}
  \caption{Performance Plot without 90/270 degree rotations}\label{fig:perf_40_41}
\end{figure}


After this experiment, we improved the column-wise access by rotating the range
block instead of the domain block for the 90/270 degree rotations. The improved
vectorized implementation has a performance roughly eight times as high as the
baseline and about twice the performance of the scalar optimized version. The
runtime was also reduced especially for large images.

\section{Conclusions}

With scalar optimizations, our implementation gains a performance speedup
of roughly 4x, whereas the vectorized implementaiton has a performance speedup
of roughly 8x compared with our reasonable straightforward implementation. It is important
to mention that the runtime could be decreased significantly and tangibly. Compressing
an image with $2048 \times 2048$ pixels took about 2 hours with the baseline implementation,
whereas the vectorized implementation needed 2.5 minutes.

With our best implementation which performs at 4 flops/cycle, we are still 4 times 
below the theoretical peak performance of 16 flops/cycle. Because the algorithm
in this form is not memory bound, we acknowledge that further performance improvements
can be expected.

Another challenge is to implement the optimizations for non-squared images whose
width and height are not powers of two, because most images in practice do not 
conform to these simplyfying assumptions.

To furthermore boost performance one would also need to apply algorithmic changes. 
Using exhaustive block mapping with a rather large
domain block pool (e.g. with four rotations) is a significant performance
bottleneck which does not necessarily lead to significant better compression results.

We consider our optimizations to be applicable in all of these more advanced
and mature fractal image compression schemes.
\section{Contributions of Team Members}

\mypar{Jonas} Focused on the baseline implementation and the following scalar optimizations 
(ILP, removal of structs, removal of pointer chasing and decreasing of memory reallocation). 
Helped in prototyping SIMD (especially the four-domain-blocks-a-time approach) 
and contributed to the solution which increased SIMD performance regarding 90 and 270 degree rotations. \\
Wrote a tool to simulate cache access in order to determine tighter memory bounds in the roofline plot. \\
Experimented with block-wise domain/range block traversal.

\mypar{Pascal} \dots

\mypar{Fabian} \dots

\notefabian{Should we mention Janis?}


% References should be produced using the bibtex program from suitable
% BiBTeX files (here: bibl_conf). The IEEEbib.bst bibliography
% style file from IEEE produces unsorted bibliography list.
% -------------------------------------------------------------------------
\bibliographystyle{IEEEtran}
\bibliography{bibl_conf}

\end{document}
