\section{Background on the Algorithm/Application}\label{sec:background}

Give a short, self-contained summary of necessary
background information on the algorithm or application that you then later optimize including a cost analysis.

For example, assume you present an
implementation of FFT algorithms. You could organize into DFT
definition, FFTs considered, and cost analysis. The goal of the
background section is to make the paper self-contained for an audience
as large as possible. As in every section
you start with a very brief overview of the section. Here it could be as follows: In this section
we formally define the discrete Fourier transform, introduce the algorithms we use
and perform a cost analysis.

\mypar{Discrete Fourier Transform}
Precisely define the transform so I understand it even if I have never
seen it before.

\mypar{Fast Fourier Transforms}
Explain the algorithm you use.

\mypar{Cost Analysis}
First define you cost measure (what you count) and then compute or determine on other ways the
cost as explained in class. In the end you will likely consolidate it into one number (e.g., adds/mults/comparisons) but be aware of major imbalances as they affect the peak performance..

Also state what is known about the complexity (asymptotic usually)
about your problem (including citations).